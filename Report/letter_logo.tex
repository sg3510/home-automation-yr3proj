% This LaTeX file uses the class impletter.cls to format
% the new Imperial College letter to be printed on logo only paper.
% LdC Foulkes Jan. 2003.

\documentclass[logo]{impletter}

%%%%%%%%%%%%%%%% Adjustment Parameters %%%%%%%%%%%%%%%%%%%%%%%%%%%
% The letter has to have have the following (measured from the top):
%	The bottom of the date at 51mm
%	The bottom of the 1st address line at 63mm
%	The bottom of the salutation at 105mm
%	The bottom of the department line (top right) at 12mm
% 	The bottom of the first line of text on the second page at 51mm
%	The margin at the left of the text should be 25mm
%	The sender's detail (top right block) should be 90mm from the
%	right edge.

% The following are adjustable parameters to account for printer
% differences. The values can be given in mm or pt
% (a point is about .35 of a mm).

% To move the printed page up (minus value) or down (plus value):
\addtolength{\topmargin}{0mm}
% To move the printed text to the right (plus) or left (minus).
\addtolength{\evensidemargin}{0pt}
\addtolength{\oddsidemargin}{0pt}
%%%%%%%%%%%%%%%%%%%%%%%%%%%%%%%%%%%%%%%%%%%%%%%%%%%%%%%%%%%%%%%%%%

\begin{document}

\headers{
% Replace with the addressee details but DO NOT remove the \\ even if empty
Addressee Name\\
Address Line 1\\
Address Line 2\\
Address Line 3\\
Address Line 4\\
}
% Replace with the salutation for this letter
{
Dear Sir or Madam,
}
% Replace with your department, address, telephone, fax and e-mail.
% Do not remove the blank lines.
{
Department of Aeronautics,\\
Imperial College London\\

% Could add a line with the room number and building here
% and end the new line with \\
South Kensington Campus, London SW7 2AZ, UK\\
Telephone +44 (0)20 7594 5048 Fax +44 (0)20 7584 8120\\

l.foulkes@imperial.ac.uk\\
www.imperial.ac.uk/aeronautics\\
}
% Replace with your name
{
Dr Letty Foulkes
}
% Replace with your qualifications
{
BA, MS, PhD, CPhys, MinstP\\
}
% Replace with your job title. A second line can be added if necessary.
{
Computer Resources Manager\\
}
\informal

% The text of the letter starts here
The name computer comes from ``to compute'' meaning to
calculate. Computers were developed originally to do the maths humans
were too slow to do. The technological development that allowed the
birth of modern computers is the transistor. Transistor devices could
be made to hold a given voltage, a high or a low. Hence, if we connect
many transistors we can represent a series made up of 0 (low) and 1
(high). The binary number system can represent any number in 1s and
0s, which means that transistors can represent any number (or a code
made up of binary numbers).

To do a calculation two things are needed: a place to hold the
numbers, the operation and the result, and a brain to do it. Inside a
computer, the first is the memory, and the second is the Central
Processor Unit(CPU). The CPU and memory must communicate to exchange
information. This communication is done via the bus. Also, the
computer needs to receive the information from somewhere and it needs
to send the result somewhere. These two operations are called INPUT
and OUTPUT. This very simple model still holds today.

The name computer comes from ``to compute'' meaning to
calculate. Computers were developed originally to do the maths humans
were too slow to do. The technological development that allowed the
birth of modern computers is the transistor. Transistor devices could
be made to hold a given voltage, a high or a low. Hence, if we connect
many transistors we can represent a series made up of 0 (low) and 1
(high). The binary number system can represent any number in 1s and
0s, which means that transistors can represent any number (or a code
made up of binary numbers).

To do a calculation two things are needed: a place to hold the
numbers, the operation and the result, and a brain to do it. Inside a
computer, the first is the memory, and the second is the Central
Processor Unit(CPU). The CPU and memory must communicate to exchange
information. This communication is done via the bus. Also, the
computer needs to receive the information from somewhere and it needs
to send the result somewhere. These two operations are called INPUT
and OUTPUT. This very simple model still holds today.

One important element that is missing in this simple picture is
timing. Pulses of voltages (high or low) can change state at a certain
rate. This rate determines how fast the computer can ``think''. So
computers must have a clock. The clock speed is what computer
manufacturers quote when selling computers (in Megahertz). This rate
is never achieved in practice by the CPU, and much less by the
transfer of information between memory and disk. Hence, the rate
should be taken only as a guide but not as the actual speed of the
computer.

% Leave the name field blank if you don't want your name to be
% printed at the bottom, but do not remove the {}.
\close{
Yours sincerely,
}
{
Letty Foulkes
}
\end{document}
